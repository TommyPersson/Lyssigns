\documentclass[a5paper]{article}

\usepackage{mall}

\begin{document}

   \computerdescription
       {Fritz}
       {CPU-server}
       {ThinLinc-server \linebreak (grafisk fjärrinloggning)}
       {CentOS 5.3}
       {x86/PC}
       {4 $\times$ 700MHz Xeon}
       {4GB}
       {Compaq DL580}
       {derfian}

   \othernotes {ThinLinc} { ThinLinc är en Linuxbaserad
     terminalserverlösning från Cendio AB. Det beter sig ungefär som
     screen, fast för din grafiska skrivbordsmiljö. Cendio har varit
     snälla nog att donera licenser för 100 samtidiga användare - dvs,
     vi kommer slå i serverns kapacitetstak före licenserna tar
     slut. För att använda ThinLinc laddar man ner sin klient från
     \texttt{http://www.cendio.com/downloads/clients/} och pekar klienten mot
     \texttt{thinlinc.lysator.liu.se}. Inloggningsnamn och lösenord är ditt
     vanliga inloggningslösenord på Lysator. Frågor och fel kan
     meddelas till ansvarig root.}


   \sshfingerprintheading
   \begin{sshfingerprint}
SSH2 RSA: 1a:f1:8a:aa:5f:a1:d9:6f:7d:3d:8e:48:16:6d:6b:37
SSH2 DSA: 47:99:0d:47:92:a4:12:98:23:04:de:5c:cd:6a:d5:88
   \end{sshfingerprint}


\end{document}
